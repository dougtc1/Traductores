\documentclass[a4paper, 10pt]{article}
\usepackage[english]{babel}
\usepackage[utf8]{inputenc} 
\usepackage{ragged2e}
\usepackage{tikz}
\usepackage{forest}
\usetikzlibrary{arrows}
\usetikzlibrary{automata, positioning}
\usepackage{lipsum}

\newcommand\blfootnote[1]{%
  \begingroup
  \renewcommand\thefootnote{}\footnote{#1}%
  \addtocounter{footnote}{-1}%
  \endgroup
}

\title{Etapa 2: Análisis Sintáctico}
\date{Enero - Marzo 2016}
\author{Benjamin Amos, Douglas Torres}

\begin{document}
	
	\maketitle
	\pagenumbering{gobble}
	\newpage
	\pagenumbering{arabic}
	
	\section{Detalles de Implementacion}
		
		\par	
		\medskip	
		Para la implementación de la segunda etapa del proyecto, utilizamos
		una herramienta, parte de la librería ply, llamada \textit{yacc}, la cual
		permite el diseño de una gramática para la creación del parser para nuestra
		lista de tokens, lo cual permitirá el diseño del árbol sintáctico abstracto, como
		finalidad de esta etapa.
		
		\par
		\medskip
		La implementacion del parser fue dividida en dos partes, un archivo para la creación
		de la gramatica libre de contexto la cual estudiara la sintaxis de un programa
		escrito en el lenguaje de estudio \textbf{BOT} y generara todas las cadenas posibles
		permitidas por el mismo, y otro archivo en el cual se almacenan las estructuras de datos
		utilizadas para la creación del árbol sintáctico abstracto, al igual que los métodos que permiten
		proporcionar la interfaz pedida.
		
		\subsection{parser.py}
		
			\par
			\medskip
			En este archivo se encuentran las reglas gramaticales permitidas por el lenguaje \textbf{BOT}.
			De este modo, se pueden producir las distintas cadenas que acepta el mismo lenguaje. Así mismo, 
			en las reglas, se encuentra la inicialización y agregación de nodos al árbol sintáctico abstracto.
			
		\subsection{arboles.py}
		
			\par
			\medskip
			Este archivo contiene las estructuras de datos utilizadas para la creación del árbol sintáctico 
			abstracto, las cuales se mencionan a continuación:
			
			\begin{enumerate}
				\item \textbf{ArbolInstr}:
				\begin{itemize}
					\item \textit{CondicionalIf}
					\item \textit{IteracionIndef}
					\item \textit{Activate}
					\item \textit{Deactivate}
					\item \textit{Advance}					
				\end{itemize}							
				\item \textbf{ArbolBin}
				\item \textbf{ArbolUn}
				\item \textbf{Ident}
				\item \textbf{Bool}
				\item \textbf{Numero}\\
			\end{enumerate}
			
			Notemos que solo se encuentran creadas estructuras de datos para el manejo de instrucciones de
			controlador de \textbf{BOT}.
	
	\newpage			
	\section{Sección Teórico-Práctica}
		
		\par
		\medskip
		En esta sección se presenta el desarrollo y respuestas para las preguntas propuestas para esta 
		etapa.
		
		\bigskip
		\begin{enumerate}
			\item Basados en la gramática \textit{G1} dada:
			\begin{enumerate}
				\item Para demostrar que la frase $\textit{NUM} + \textit{NUM} + \textit{NUM}$ es ambigua,
				mostraremos que existen dos árboles de derivación distintos que representan a la frase:
				\\\
				\begin{center}
				\begin{forest}
					[\textit{E}
						[\textit{E}
							[\textit{E}
								[\textit{NUM}]
							]
							[$+$]
							[\textit{E}
								[\textit{NUM}]
							]
						]						
						[$+$]
						[\textit{E}
							[\textit{NUM}]
						]
					]
				\end{forest}
				\qquad
				\begin{forest}
					[\textit{E}
						[\textit{E}
							[\textit{NUM}]
						]
						[$+$]
						[\textit{E}
							[\textit{E}
								[\textit{NUM}]
							]
							[$+$]
							[\textit{E}
								[\textit{NUM}]							
							]
						]
					]
				\end{forest}
				\end{center}
			Notemos entonces que la frase en cuestión es ambigua.\\
			
			\item Sean \textit{Izq}$($\textit{G1}$)$ y \textit{Der}$($\textit{G1}$)$ definidas 
			de la siguiente manera:\\
			\begin{itemize}
				\item \textit{Izq}$($\textit{G1}$)$\emph{:} \textit{Expr} $\rightarrow$ \textit{Expr} $+$ \textit{Expr$^{'}$}\\
				$\hphantom{\quad}\hspace{61pt}|$ \, \textit{Expr$^{'}$}\\ $\hphantom{\quad}\hspace{29pt}$ 
				\textit{Expr$^{'}$} $\rightarrow$ \textit{NUM} \\
				\item \textit{Der}$($\textit{G1}$)$\emph{:} \textit{Expr} $\rightarrow$ \textit{Expr$^{'}$} $+$ \textit{Expr}\\
				$\hphantom{\quad}\hspace{61pt}|$ \, \textit{Expr$^{'}$}\\ $\hphantom{\quad}\hspace{29pt}$ 
				\textit{Expr$^{'}$} $\rightarrow$ \textit{NUM}\\
			\end{itemize}
			
			\item En efecto, si importa la forma en la que se asocian las expresiones en esta gramática. Veamos con
			un ejemplo por qué es relevante. Supongamos que el alfabeto acepta los operadores $-$ y $\div$. Sea la expresión
			$\textit{NUM} - \textit{NUM} \div \textit{NUM}$. Construyamos los árboles de derivación respectivos a 
			\textit{Izq}$($\textit{G1}$)$ y \textit{Der}$($\textit{G1}$)$ respectivamente:\\
			\begin{center}
			\begin{forest}
					[\textit{E}
						[\textit{E}
							[\textit{E}
								[\textit{NUM}]
							]
							[$-$]
							[\textit{E}
								[\textit{NUM}]
							]
						]						
						[$\div$]
						[\textit{E}
							[\textit{NUM}]
						]
					]
				\end{forest}		
				\qquad	
				\begin{forest}
					[\textit{E}
						[\textit{E}
							[\textit{NUM}]
						]
						[$-$]
						[\textit{E}
							[\textit{E}
								[\textit{NUM}]
							]
							[$\div$]
							[\textit{E}
								[\textit{NUM}]							
							]
						]
					]
				\end{forest}
			\end{center}
			Notemos que \textit{Izq}$($\textit{G1}$)$ generaría a la expresión $(\textit{NUM} - \textit{NUM}) \div \textit{NUM}$,
			mientras que \textit{Der}$($\textit{G1}$)$ genera a $\textit{NUM} - (\textit{NUM} \div \textit{NUM})$
			
			\end{enumerate}		
			
		\item Basados en la gramática \textit{G2}, tenemos que:
		\begin{enumerate}
			\item Se entiende que la gramática \textit{G2} tiene los mismos problemas de ambigüedad que \textit{G1} ya que, 
			\textit{Expr} $\equiv$ \textit{Instr}, $+$ $\equiv$ $;$ y \textit{NUM} $\equiv$ \textit{IS}. Luego, existen dos o más 
			arboles de derivación diferentes para una frase que reconoce la gramática. Las únicas frases no ambiguas de la gramática
			\textit{G2} son la frase \textit{IS} y la frase \textit{IS} ; \textit{IS} ya que sólo hay un árbol para construirlas.
				
		\item Dado que la gramática en este caso reconoce secuencias de instrucciones y no operaciones de expresiones,
		se puede afirmar que no importa el sentido en el que se asocien las expresiones. Para esto, supongamos que existen dos
		gramáticas \textit{Izq}$($\textit{G2}$)$ y \textit{Der}$($\textit{G2}$)$ tales que:\\
		\begin{itemize}
				\item \textit{Izq}$($\textit{G2}$)$\emph{:} \textit{Instr} $\rightarrow$ \textit{Instr} $;$ \textit{Inst$^{'}$}\\
				$\hphantom{\quad}\hspace{61pt}|$ \, \textit{Instr$^{'}$}\\ $\hphantom{\quad}\hspace{29pt}$ 
				\textit{Instr$^{'}$} $\rightarrow$ \textit{IS} \\
				\item \textit{Der}$($\textit{G2}$)$\emph{:} \textit{Instr} $\rightarrow$ \textit{Instr$^{'}$} $;$ \textit{Instr}\\
				$\hphantom{\quad}\hspace{61pt}|$ \, \textit{Expr$^{'}$}\\ $\hphantom{\quad}\hspace{29pt}$ 
				\textit{Instr$^{'}$} $\rightarrow$ \textit{IS}\\
			\end{itemize}
		Dado que los árboles de derivación que se presentan son lo mismos que los de \textit{G1}, notemos que las expresiones que
		se generarían son $(\textit{IS}\, ; \textit{IS})\, ; \textit{IS}$ y $\textit{IS}\, ; (\textit{IS}\,	 ; \textit{IS})$. Siendo
		éstas instrucciones, no importa el orden en el que se realicen.
	
		\item Veamos una derivación más a la izquierda para la frase $\textit{IS}\, ; \textit{IS}\, ; \textit{IS}$:
		\begin{center}
			\textit{Instr} $\Rightarrow$ \textit{Instr} ; \textit{Instr} $\Rightarrow$ \textit{Instr} ; \textit{Instr} ; \textit{Instr} 
			$\Rightarrow$ \textit{Instr} ; \textit{Instr} ; \textit{IS} $\Rightarrow$ \textit{Instr} ; \textit{IS} ; \textit{IS} 
			$\Rightarrow$ \textit{IS} ; \textit{IS} ; \textit{IS}\\
		\end{center}
		Veamos una derivación más a la derecha para la misma frase:\\
		\begin{center}
			\textit{Instr} $\Rightarrow$ \textit{Instr} ; \textit{Instr} $\Rightarrow$ \textit{Instr} ; \textit{Instr} ; \textit{Instr} 
			$\Rightarrow$ \textit{IS} ; \textit{Instr} ; \textit{Instr} $\Rightarrow$ \textit{IS} ; \textit{IS} ; \textit{Instr} 
			$\Rightarrow$ \textit{IS} ; \textit{IS} ; \textit{IS}\\
		\end{center}
		
		Para ilustrarlo, se muestran dos árboles de derivación diferentes que generan la misma frase:\\
		\begin{center}
			\begin{forest}
				[Instr
					[Instr
						[Instr
							[\textit{IS}]
						]
						[$;$]
						[Instr
							[\textit{IS}]
						]
					]						
					[$;$]
					[Instr
						[\textit{IS}]
					]
				]
			\end{forest}
			\qquad
			\begin{forest}
				[\textit{Instr}
					[\textit{Instr}
						[\textit{IS}]
					]
					[$;$]
					[\textit{Instr}
						[\textit{Instr}
							[\textit{IS}]
						]
						[$;$]
						[\textit{Instr}
							[\textit{IS}]							
						]
					]
				]
			\end{forest}
			\end{center}
		\end{enumerate}
		
	\item Basados en la gramatica \textit{G3}:
	\begin{enumerate}
	\item
	Mostraremos mediante dos árboles de derivación que la frase \textit{IF Bool} : \textit{IS} ; \textit{IS} es ambigua:
	\begin{center}
			\begin{forest}
				[\textit{Instr}
					[\textit{IF}]
					[\textit{Bool}]
					[\textbf{:}]
					[\textit{Instr}
						[\textit{Instr}
							[\textit{IS}]						
						]
						[$;$]
						[\textit{Instr}
							[\textit{IS}]
						]					
					]					
				]
			\end{forest}
			\qquad
			\begin{forest}
				[\textit{Instr}
					[\textit{Instr}
						[\textit{IF}]
						[\textit{Bool}]
						[\textbf{:}]
						[\textit{Instr}
							[\textit{IS}]						
						]
					]
					[$;$]
					[\textit{Instr}
						[\textit{IS}]
					]				
				]
			\end{forest}
		\end{center}				
	\item Una frase \textit{g} de \textit{G3} sin ocurrencias de ; que sea ambigua puede ser:\\
	\begin{center}
		\textit{IF Bool $:$ IF Bool $:$ IF Bool $:$ IS ELSE $:$ IS}\\
	\end{center}	
			
	\item Para mostrar que es ambigua, mostraremos dos árboles de derivación distintos:\\
	\begin{center}
			\begin{forest}
				[\textit{Instr}
					[\textit{IF}]
					[\textit{Bool}]
					[$:$]
					[\textit{Instr}
						[\textit{Instr}
							[\textit{IF}]
							[\textit{Bool}]
							[$:$]
							[\textit{Instr}
								[\textit{IF}]
								[\textit{Bool}]
								[$:$]
								[\textit{Instr}
									[\textit{IS}]								
								]						
							]						
						]					
					]
					[\textit{ELSE}]
					[$:$]
					[\textit{Instr}
						[\textit{IS}]					
					]				
				]
			\end{forest}
			\qquad
			\begin{forest}
				[\textit{Instr}
					[\textit{IF}]
					[\textit{Bool}]
					[$:$]
					[\textit{Instr}
						[\textit{IF}]
						[\textit{Bool}]
						[$:$]
						[\textit{Instr}
							[\textit{IF}]
							[\textit{Bool}]
							[$:$]
							[\textit{Instr}
								[\textit{IS}]							
							]			
						]
						[\textit{ELSE}]
						[$:$]
						[\textit{Instr}
							[\textit{IS}]						
						]
					]				
				]
			\end{forest}
		\end{center}
	
	\item Utilizando llaves, podríamos escribir:
	\begin{itemize}
		\item Dos interpretaciones de la frase \textit{f} con \{ y \}:
		\begin{itemize}
		\item \textit{IF Bool} $\{$ \textit{IS} ; \textit{IS} $\}$
		\item \textit{IF Bool} $\{$ \textit{IS} $\}$ ; \textit{IS}
		\end{itemize}
		\newpage
		\item Dos interpretaciones de la frase \textit{g} con \{ y \}:
		\begin{itemize}
		\item \textit{IF Bool $\{$ IF Bool $\{$ IF Bool $\{$ IS $\}$ $\}$ $\}$ ELSE $\{$ IS $\}$}
		\item \textit{IF Bool $\{$ IF Bool $\{$ IF Bool $\{$ IS $\}$ ELSE $\{$ IS $\}$ $\}$ $\}$}
		\end{itemize}
		
	\end{itemize}
	
	\item Utilizando el terminador \textit{end}, podemos escribir:
	\begin{itemize}
		\item Dos interpretaciones de la frase \textit{f} con \{ y \}:
		\begin{itemize}
		\item \textit{IF Bool} $:$ \textit{IS} ; \textit{IS} \textit{end}
		\item \textit{IF Bool} $:$ \textit{IS} \textit{end} ; \textit{IS}
		\end{itemize}

		\item Dos interpretaciones de la frase \textit{g} con \{ y \}:
		\begin{itemize}
		\item \textit{IF Bool $:$ IF Bool $:$ IF Bool $:$ IS \textit{end} \textit{end} \textit{end} ELSE $:$ IS \textit{end}}
		\item \textit{IF Bool $:$ IF Bool $:$ IF Bool $:$ IS \textit{end} ELSE $:$ IS \textit{end} \textit{end} \textit{end}}
		\end{itemize}
		
	\end{itemize}
	
	\end{enumerate}
	\end{enumerate}
	\blfootnote{Benjamin Amos \#12-10240, Douglas Torres \#11-11027 / Enero - Marzo 2016}				 		
	
			
\end{document}
